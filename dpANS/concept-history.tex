%-*- Mode: TeX -*-
%%Scope, Purpose, and History
\beginsubSection{Scope and Purpose}
The specification set forth in this document is designed to promote
the portability of \clisp\ programs among a variety of data processing
systems. It is a language specification aimed at an audience of
implementors and knowledgeable programmers. It is neither a tutorial nor
an implementation guide.
\endsubSection%{Scope and Purpose}

\beginsubSection{History}
 
Lisp is a family of languages with a long history.  Early key ideas in
Lisp were developed by John McCarthy during the 1956 Dartmouth Summer
Research Project on Artificial Intelligence.  McCarthy's motivation
was to develop an algebraic list processing language for artificial
intelligence work.
Implementation efforts for early dialects of Lisp were undertaken on
the IBM~704, the IBM~7090, the Digital Equipment Corporation (DEC) PDP-1,
the DEC~PDP-6, and the PDP-10. The primary dialect of Lisp between
1960 and 1965 was Lisp~1.5. By the early 1970's there were two
predominant dialects of Lisp, both arising from these early efforts:
MacLisp and Interlisp.
For further information about very early Lisp dialects, 
see {\AnatomyOfLisp} or {\LispOnePointFive}.
 
MacLisp improved on the Lisp~1.5 notion of special variables and error
handling. MacLisp also introduced the concept of functions that could take
a variable number of arguments, macros, arrays, non-local dynamic
exits, fast arithmetic, the first good Lisp compiler, and an emphasis
on execution speed. 
% This next sentence transplanted from a later paragraph about PSL.
% JonL says: My recollection was that it was closer to 100 than to 50.
%            More likely, it had been received by about 100 sites, 
%            and was actually in continued usage by 50 at the end of the decade.
% KMP: Changed "was available" to "was in use" to compensate. (6-Dec-91)
By the end of the 1970's, MacLisp was in use at over 50 sites.
%From about 1969 onward Jonl White was the dominant force behind Maclisp.
For further information about Maclisp, 
see {\Moonual} or {\Pitmanual}.
 
Interlisp introduced many ideas into Lisp programming environments and
methodology. One of the Interlisp ideas that influenced \clisp\ was an iteration
construct implemented by Warren Teitelman that inspired the \macref{loop}
macro used both on the Lisp Machines and in MacLisp, and now in \clisp.
For further information about Interlisp,
see {\InterlispManual}.
 
Although the first implementations of Lisp were on the IBM~704 and the
IBM~7090, later work focussed on the DEC
PDP-6 and, later, PDP-10 computers, the latter being the mainstay of
Lisp and artificial intelligence work at such places as 
Massachusetts Institute of Technology (MIT), Stanford University,
and 
Carnegie Mellon University (CMU) from the mid-1960's through much of the 1970's.
The PDP-10 computer and its predecessor the PDP-6 computer were, by
design, especially well-suited to Lisp because they had 36-bit words
and 18-bit addresses. This architecture allowed a \term{cons} cell to be
stored in one word; single instructions could extract the 
\term{car} and \term{cdr}
parts.  The PDP-6 and PDP-10 had fast, powerful stack instructions
that enabled fast function calling.
But the limitations of the PDP-10 were evident by 1973: it supported a
small number of researchers using Lisp, and the small, 18-bit address
space ($2^{18}$ $=$ 262,144 words) limited the size of a single
program.
One response to the address space problem was the Lisp Machine, a
special-purpose computer designed to run Lisp programs.  The other
response was to use general-purpose computers with address spaces
larger than 18~bits, such as the DEC VAX and
the \hbox{S-1}~Mark~IIA.
For further information about S-1 Common Lisp, see ``{\SOneCLPaper}.''
 
The Lisp machine concept was developed in the late 1960's.  In the
early 1970's, Peter Deutsch, working with 
Daniel Bobrow, implemented a Lisp on the
Alto, a single-user minicomputer, using microcode to interpret a
byte-code implementation language. Shortly thereafter, Richard
Greenblatt began work on a different hardware and instruction set
design at MIT.
Although the Alto was not a total success as a Lisp machine, a dialect
of Interlisp known as Interlisp-D became available on the D-series
machines manufactured by Xerox---the Dorado, Dandelion,
Dandetiger, and Dove (or Daybreak).
An upward-compatible extension of MacLisp called Lisp
Machine Lisp became available on the early MIT Lisp Machines.
Commercial Lisp machines from Xerox, Lisp Machines (LMI), and
Symbolics were on the market by 1981.
For further information about Lisp Machine Lisp, see {\Chinual}.
 
During the late 1970's, Lisp Machine Lisp began to expand towards a
much fuller language.  Sophisticated lambda lists, 
\f{setf}, multiple values, and structures
like those in \clisp\ are the results of early
experimentation with programming styles by the Lisp Machine group.
Jonl White and others migrated these features to MacLisp.
Around 1980, Scott Fahlman and others at CMU began work on a Lisp to
run on the Scientific Personal Integrated Computing
Environment (SPICE) workstation.  One of the goals of the project was to
design a simpler dialect than Lisp Machine Lisp.
 
The Macsyma group at MIT began a project during the late 1970's called
the New Implementation of Lisp (NIL) for the VAX, which was headed by
White.  One of the stated goals of the NIL project was to fix many of
the historic, but annoying, problems with Lisp while retaining significant 
compatibility with MacLisp.  At about the same time, a research group at
Stanford University and Lawrence Livermore National Laboratory headed
by Richard P. Gabriel began the design of a Lisp to run on the
\hbox{S-1}~Mark~IIA supercomputer.  \hbox{S-1}~Lisp, never completely
functional, was the test bed for adapting advanced compiler techniques
to Lisp implementation.  Eventually the \hbox{S-1} and NIL groups
collaborated.
For further information about the NIL project,
see ``{\NILReport}.''
 
% The first efforts towards Lisp standardization were Standard Lisp and
% Portable Standard Lisp (PSL).  In 1969, Anthony Hearn and Martin Griss
% defined Standard Lisp---a subset of Lisp~1.5 and other dialects---to
% transport REDUCE, a symbolic algebra system. 
% PSL was designed to provide more control over the environment and
% the compiler.
% At the end of the 1970's, PSL ran on about a dozen different
% computers.  PSL and Franz Lisp---a MacLisp-like dialect for Unix
% machines---were the first examples of widely available Lisp dialects
% on multiple hardware platforms. MacLisp was available at over 50
% sites.
% For further information about Standard Lisp, 
% see ``{\StandardLispReport}.''
%
%% Sandra:  This paragraph contains inaccuracies and is not well-organized.
%%          Suggested rewrite [taken -kmp] below.

The first effort towards Lisp standardization was made in 1969, 
when Anthony Hearn and Martin Griss at the University of Utah 
defined Standard Lisp---a subset of Lisp~1.5 and other dialects---to 
transport REDUCE, a symbolic algebra system.
During the 1970's, the Utah group implemented first a retargetable
optimizing compiler for Standard Lisp,
and then an extended implementation known as Portable Standard Lisp (PSL).
By the mid 1980's, PSL ran on about a dozen kinds of computers.
For further information about Standard Lisp, see ``{\StandardLispReport}.''
 
PSL and Franz Lisp---a MacLisp-like dialect for Unix machines---were 
the first examples of widely available Lisp dialects on multiple 
hardware platforms. 

One of the most important developments in Lisp occurred during the
second half of the 1970's: Scheme. Scheme, designed by Gerald J.
Sussman and Guy L. Steele Jr., is a simple dialect of Lisp whose
design brought to Lisp some of the ideas from programming language
semantics developed in the 1960's.  Sussman was one of the prime
innovators behind many other advances in Lisp technology from the late
1960's through the 1970's.
The major contributions of Scheme were lexical scoping, lexical
closures, first-class continuations, and simplified syntax (no
separation of value cells and function cells). Some of these contributions made
a large impact on the design of \clisp.
For further information about Scheme, see {\IEEEScheme} or ``{\RevisedCubedScheme}.''

In the late 1970's object-oriented programming concepts started to
make a strong impact on Lisp. 
At MIT, certain ideas from Smalltalk made their way into several
widely used programming systems.
Flavors, an object-oriented programming system with multiple inheritance, 
was developed at MIT for the Lisp machine community by Howard Cannon and others.
At Xerox, the experience with Smalltalk and 
Knowledge Representation Language (KRL) led to the development of 
Lisp Object Oriented Programming System (LOOPS) and later Common LOOPS.
For further information on Smalltalk, see {\SmalltalkBook}.
For further information on Flavors, see {\FlavorsPaper}.

These systems influenced the design of the Common Lisp Object System (CLOS).
CLOS was developed specifically for this standardization effort,
and was separately written up in ``\CLOSPaper.''  However, minor details
of its design have changed slightly since that publication, and that paper 
should not be taken as an authoritative reference to the semantics of the
\CLOS\ as described in this document.

In 1980 Symbolics and LMI were developing Lisp Machine Lisp; stock-hardware
implementation groups were developing NIL, Franz Lisp, and PSL; Xerox
was developing Interlisp; and the SPICE project at CMU was developing
a MacLisp-like dialect of Lisp called SpiceLisp.
 
In April 1981, after a DARPA-sponsored meeting concerning the
splintered Lisp community, Symbolics, the SPICE project, the NIL
project, and the \hbox{S-1}~Lisp project joined together to define
\clisp.  Initially spearheaded by White and Gabriel, the
driving force behind this grassroots effort was provided by Fahlman,
Daniel Weinreb, David Moon, Steele,  and Gabriel.
\clisp\ was designed as a description of a family of languages.  The
primary influences on \clisp\ were Lisp Machine Lisp, MacLisp, NIL,
\hbox{S-1}~Lisp, Spice Lisp, and Scheme.
\CLtL\ is a description of that design.  Its
semantics were intentionally underspecified in places where it was
felt that a tight specification would overly constrain \clisp\
research and use.
%% Per X3J13. -kmp 05-Oct-93
%Between 1984 and 1989, \clisp\ became a de facto standard. 

In 1986 X3J13 was formed as a technical working group to
produce a draft for an ANSI \clisp\ standard. Because of the
acceptance of \clisp, the goals of this group differed from those of
the original designers. These new goals included stricter
standardization for portability, an object-oriented programming
system, a condition system, iteration facilities, and a way to handle
large character sets.  To accommodate those
goals, a new language specification, this
document, was developed.
 
\endsubSection%{History}

