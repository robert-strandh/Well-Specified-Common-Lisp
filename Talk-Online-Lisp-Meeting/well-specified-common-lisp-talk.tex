\documentclass{beamer}
\usepackage[utf8]{inputenc}
\beamertemplateshadingbackground{red!10}{blue!10}
%\usepackage{fancybox}
\usepackage{epsfig}
\usepackage{verbatim}
\usepackage{url}
%\usepackage{graphics}
%\usepackage{xcolor}
\usepackage{fancybox}
\usepackage{moreverb}
%\usepackage[all]{xy}
\usepackage{listings}
\usepackage{filecontents}
\usepackage{graphicx}

\lstset{
  language=Lisp,
  basicstyle=\scriptsize\ttfamily,
  keywordstyle={},
  commentstyle={},
  stringstyle={}}

\def\inputfig#1{\input #1}
\def\inputeps#1{\includegraphics{#1}}
\def\inputtex#1{\input #1}

\inputtex{logos.tex}

%\definecolor{ORANGE}{named}{Orange}

\definecolor{GREEN}{rgb}{0,0.8,0}
\definecolor{YELLOW}{rgb}{1,1,0}
\definecolor{ORANGE}{rgb}{1,0.647,0}
\definecolor{PURPLE}{rgb}{0.627,0.126,0.941}
\definecolor{PURPLE}{named}{purple}
\definecolor{PINK}{rgb}{1,0.412,0.706}
\definecolor{WHEAT}{rgb}{1,0.8,0.6}
\definecolor{BLUE}{rgb}{0,0,1}
\definecolor{GRAY}{named}{gray}
\definecolor{CYAN}{named}{cyan}

\newcommand{\orchid}[1]{\textcolor{Orchid}{#1}}
\newcommand{\defun}[1]{\orchid{#1}}

\newcommand{\BROWN}[1]{\textcolor{BROWN}{#1}}
\newcommand{\RED}[1]{\textcolor{red}{#1}}
\newcommand{\YELLOW}[1]{\textcolor{YELLOW}{#1}}
\newcommand{\PINK}[1]{\textcolor{PINK}{#1}}
\newcommand{\WHEAT}[1]{\textcolor{wheat}{#1}}
\newcommand{\GREEN}[1]{\textcolor{GREEN}{#1}}
\newcommand{\PURPLE}[1]{\textcolor{PURPLE}{#1}}
\newcommand{\BLACK}[1]{\textcolor{black}{#1}}
\newcommand{\WHITE}[1]{\textcolor{WHITE}{#1}}
\newcommand{\MAGENTA}[1]{\textcolor{MAGENTA}{#1}}
\newcommand{\ORANGE}[1]{\textcolor{ORANGE}{#1}}
\newcommand{\BLUE}[1]{\textcolor{BLUE}{#1}}
\newcommand{\GRAY}[1]{\textcolor{gray}{#1}}
\newcommand{\CYAN}[1]{\textcolor{cyan }{#1}}

\newcommand{\reference}[2]{\textcolor{PINK}{[#1~#2]}}
%\newcommand{\vect}[1]{\stackrel{\rightarrow}{#1}}

% Use some nice templates
\beamertemplatetransparentcovereddynamic

\newcommand{\A}{{\mathbb A}}
\newcommand{\degr}{\mathrm{deg}}

\title{Well Specified \commonlisp{} (WSCL)}

\author{Robert Strandh}
\date{October, 2021}

%\inputtex{macros.tex}


\begin{document}
\frame{
\titlepage
\vfill
\small{Online Lisp Meeting}
}

\setbeamertemplate{footline}{
\vspace{-1em}
\hspace*{1ex}{~} \GRAY{\insertframenumber/\inserttotalframenumber}
}

\frame{
\frametitle{Motivation}
\vskip 0.25cm
\begin{itemize}
\item We would like for \commonlisp{} to be a \emph{safe} language, at
  least when the \texttt{safety} optimize quality is $3$, also known
  as \emph{safe code}.
\item But the standard contains many instances of unspecified or
  undefined behavior.
\item Some such instances are justified.  Most are not.
\item Most current \commonlisp{} implementations define this behavior
  all in the nearly same way, and they all define the behavior to be
  safe in \emph{safe code}.
\end{itemize}
}

\frame{
\frametitle{Example: \texttt{aref}}
\vskip 0.25cm
In the ``Arguments and Values'' section of the dictionary entry, the
standard says about the first argument: ``\textit{array} -- an \textit{array}''.
\vskip 0.25cm
In the ``Exceptional Situations'' section, it says: ``None''.
\vskip 0.25cm
So what happens if \textit{array} is not an \textit{array}, say if
someone calls \texttt{(aref '(a b c) 0)}?
}

\frame{
\frametitle{Example: \texttt{aref}}
\vskip 0.25cm
The answer is in section 1.4.4.3 entitled: The ``Arguments and
Values'' Section of a Dictionary Entry.
\vskip 0.25cm
``An English language description of what arguments the operator
accepts and what values it returns...''
\vskip 0.25cm
``Except as explicitly specified otherwise, the consequences are
undefined if these type restrictions are violated.''
}

\frame{
\frametitle{Well Specified \commonlisp{} (WSCL)}
https://github.com/robert-strandh/Well-Specified-Common-Lisp
https://github.com/s-expressionists/wscl
\vskip 0.25cm
Objectives:
\vskip 0.25cm
\begin{itemize}
\item Specify behavior that is unspecified or undefined as much as
  possible, given behavior of current implementations.
\item Publish the result as a supplement to the \commonlisp{}
  standard.
\item Create a test suite as a supplement to the existing ANSI test
  suite.
\end{itemize}
}

\frame{
\frametitle{Current state}
\begin{enumerate}
\item We have around $40$ instances of undefined or unspecified
  behavior, each one with a proposal for a specification.
\item These instances are written up according to the X3J13 ``issue''
  format.
\item Thanks to the work of Jan Moringen, a version of the draft ANSI
  proposal can be created that allows these proposals to be viewed
  when the pointer is hovered over the relevant dictionary entry.
\end{enumerate}
}

\frame{
\frametitle{Future work}
}

\frame{
  \frametitle{Acknowledgments}
}

\frame{
\frametitle{Thank you}
}

%% \frame{\tableofcontents}
%% \bibliography{references}
%% \bibliographystyle{alpha}

\end{document}
