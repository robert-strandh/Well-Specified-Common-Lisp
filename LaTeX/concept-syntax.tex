../dpANS/concept-syntax.tex